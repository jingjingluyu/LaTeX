%-*- coding: UTF-8 -*-
\documentclass[UTF8]{ctexart}
\usepackage{geometry}
\geometry{a4paper,centering,scale=0.8}
\usepackage[gen]{eurosym}

\title{\heiti Chapter 2 Typesetting Text}
\author{ Donald E. Knuth(高德纳)}

\begin{document}
\maketitle
\section{文章和语言的结构}
书写一篇文章最重要的一点是把想法、信息、知识传达给读者。 \LaTeX 与其他类型的排版系统不同,你只需要告诉它一个文本的逻辑结构和语义结构。然后
根据文档类文件和各种样式的文件中所带有的“规则”来生成文件。

\LaTeX 中最重要的文本单位是段。分段需要在对应的源码中空一行,如果要继续写可以用换行符,用\texttt{\textbackslash}\texttt{\textbackslash}或\texttt{\textbackslash}newline。下面是说明该换行还是该另起一段的三个正确示例:

\texttt{\%} Example 1

\texttt{\textbackslash}ldots when Einstein introduced his formula

\texttt{\textbackslash}begin\texttt{\{}equation\texttt{\}}

   \qquad e = m \texttt{\textbackslash}cdot c\texttt{~\^}2 \texttt{\textbackslash} ; ,

\texttt{\textbackslash}end\texttt{\{}equation\texttt{\}}

which is at the same time the most widely known

and the least well understood physical formula.
\\

\texttt{\%} Example 2

\texttt{\textbackslash}ldots from which follows Kirchhoff' s current law:

\texttt{\textbackslash}begin\texttt{\{}equation\texttt{\}}

  \qquad \texttt{\textbackslash}sum\texttt{\_}\texttt{\{}k=1\texttt{\}}\texttt{~\^}\texttt{\{}n\texttt{\}}  I\texttt{\_}k=0 \texttt{\textbackslash};

\texttt{\textbackslash}end\texttt{\{}equation\texttt{\}}
\\

Kirchhoff' s voltage law can be derives \texttt{\textbackslash}ldots
\\

\texttt{\%} Example 3

\texttt{\textbackslash}ldots which has several advantages.

\texttt{\textbackslash}begin\texttt{\{}equation\texttt{\}}

   \qquad I\texttt{\_}D = I\texttt{\_}F - I\texttt{\_}R

\texttt{\textbackslash}end\texttt{\{}equation\texttt{\}}

is the core of a very different transistor model. \texttt{\textbackslash}ldots
\section{断行和断页}
\subsection{合理分段}
\begin{itemize}
  \item \texttt{\textbackslash}\texttt{\textbackslash} or \texttt{\textbackslash}newline:

  断行但不是另起一段。\texttt{\textbackslash}\texttt{\textbackslash}也在表格、公式等地方用于分行,而
\texttt{\textbackslash}newline只用于文本段落中。
  \item \texttt{\textbackslash}\texttt{\textbackslash}\texttt{$\ast$}:

  断行,但不另起一页和不断页。
  \item \texttt{\textbackslash}newpage or \texttt{\textbackslash}clearpage:

  通常情况下两个命令都能起到另起一页的作用,但有一些区别:一是在双排版中 \texttt{\textbackslash}newpage 只起到另起一栏的作用;二是涉及到浮动体的排版上行为不同。

  \item \texttt{\textbackslash}linebreak\texttt{[}$\langle$n$\rangle$\texttt{]} \texttt{\textbackslash}nolinebreak\texttt{[}$\langle$n$\rangle$\texttt{]} \texttt{\textbackslash}pagebreak\texttt{[}$\langle$n$\rangle$\texttt{]}
  \texttt{\textbackslash}nopagebreak\texttt{[}$\langle$n$\rangle$\texttt{]}:

  不满足于 \LaTeX 默认的断行和断页位置,用其高速哪些地方适合断页,哪些地方不合适。$\langle$n$\rangle$代表合适/不合适的程度,取值范围为 0-4,不带可选参数时,缺省为4。以上命令适合给出优先考虑断行断页\texttt{\textbackslash}禁止断行断页的位置,
  但不适合直接拿来断行或断页,使用\texttt{\textbackslash}newline 或\texttt{\textbackslash}newpage 等是更好的选择。
\end{itemize}
\subsection{连字符}
如果遇到很长的英文单词,仅在单词之间的位置断行无法生成宽度匀称的行时,就要考虑从单词中间段开。对于绝大部分单词,\LaTeX 能够找到合适的断词位置,在断开的行尾加上连字符 - 。如果一些单词没能自动断词,我们可以在单词内手动使用\texttt{\textbackslash}-命令指定断词的位置。
另外,也可以使用\texttt{\textbackslash}hyphenation\texttt{\{}word list\texttt{\}}命令来指定使用连字符的位置,例如texttt{\textbackslash}hyphenation\texttt{\{}FORTRAN Hy-phen-a-tion\texttt{\}},其中的word list是不区分大小写的。
\begin{itemize}
   \item \texttt{\textbackslash}mbox\texttt{\{}text\texttt{\}}或\texttt{\textbackslash}fbox\texttt{\{}text\texttt{\}}:

   会避免text被连字符分开。\texttt{\textbackslash}fbox\texttt{\textbackslash}mbox多了个可见的框。
\end{itemize}
\section{预定义好的字符串}
\begin{itemize}
  \item \texttt{\textbackslash}today : 打印当天日期
  \item \texttt{\textbackslash}TeX : \TeX
  \item \texttt{\textbackslash}LaTeX : \LaTeX
  \item \texttt{\textbackslash}LaTeXe : \LaTeXe
\end{itemize}
\section{特殊字符和符号}
\subsection{引号}
\begin{itemize}
  \item 双引号:``...tex...''
  \item 单引号:`...tex...'

     ``Please press the `x' key.''
\end{itemize}
\subsection{短划线和连字符}
在\LaTeX 中有下面四种横杠:
\begin{itemize}
   \item - :- 连字符,用于连接词语。

   daughter-in-law, X-rated
   \item - - :-- 短破折号,常用于连接数字表示起止范围。

   pages 13--67
   \item - - - :--- 长破折号,常用于表示意思的转换。

   yes---or no?
   \item \$-\$ :$-$ 减号

   $0$,$1$ and $-1$
\end{itemize}
\subsection{波浪线}
\begin{itemize}
    \item \texttt{\textbackslash}\texttt{~\~}\texttt{\{}\texttt{\}} :\~{}

http://www.rich.edu/\~{}bush
    \item \$\texttt{\textbackslash}sim\$ : $\sim$

http://www.clever.edu/$\sim$demo
\end{itemize}
\subsection{斜杠}
\begin{itemize}
  \item read/write: 不允许用连字符拆分
  \item read\texttt{\textbackslash}slash write: 不允许用连字符拆分
\end{itemize}
\subsection{度}
\begin{itemize}
  \item \$30\texttt{\textbackslash}, \texttt{~\^}\texttt{\{}\texttt{\textbackslash}cric\texttt{\}}\texttt{\textbackslash}mathrm\texttt{\{}C\texttt{\}}\$ :
$-30\,^{\circ}\mathrm{C}$
  \item 30 \texttt{\textbackslash}textcelsius\texttt{\{}\texttt{\}}: 30 \textcelsius{}
  \item 86 \texttt{\textbackslash}textdegree\texttt{\{}\texttt{\}}F: 86 \textdegree{}F
\end{itemize}
\subsection{欧元符号}
 首先需要在导言区加载textcomp包:

 \texttt{\textbackslash}usepackage\texttt{\{}textcomp\texttt{\}}

 使用命令输出:

 \texttt{\textbackslash}texteuro

 如果所用的字体不包含欧元符号或者想用别的字体的欧元符号,导入eurosym宏包,用gen来替换official参数可以使用和当前字体匹配的欧元符号:

 \texttt{\textbackslash}usepackage[official]\texttt{\{}eurosym\texttt{\}}
 \begin{itemize}
  \item \texttt{\textbackslash}texteuro:\texteuro
  \item \texttt{\textbackslash}euro: \euro
\end{itemize}

















\end{document}
