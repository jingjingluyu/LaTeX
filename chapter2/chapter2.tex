%-*- coding: UTF-8 -*-
\documentclass[UTF8]{ctexart}
\usepackage{geometry}
\geometry{a4paper,centering,scale=0.8}
\usepackage[gen]{eurosym}

\title{\heiti Chapter 2 Typesetting Text}
\author{ Donald E. Knuth(高德纳)}

\begin{document}
\maketitle
\section{文章和语言的结构}
书写一篇文章最重要的一点是把想法、信息、知识传达给读者。 \LaTeX 与其他类型的排版系统不同,你只需要告诉它一个文本的逻辑结构和语义结构。然后
根据文档类文件和各种样式的文件中所带有的“规则”来生成文件。

\LaTeX 中最重要的文本单位是段。分段需要在对应的源码中空一行,如果要继续写可以用换行符,用\texttt{\textbackslash}\texttt{\textbackslash}或\texttt{\textbackslash}newline。下面是说明该换行还是该另起一段的三个正确示例:

\texttt{\%} Example 1

\texttt{\textbackslash}ldots when Einstein introduced his formula

\texttt{\textbackslash}begin\texttt{\{}equation\texttt{\}}

   \qquad e = m \texttt{\textbackslash}cdot c\texttt{~\^}2 \texttt{\textbackslash} ; ,

\texttt{\textbackslash}end\texttt{\{}equation\texttt{\}}

which is at the same time the most widely known

and the least well understood physical formula.
\\

\texttt{\%} Example 2

\texttt{\textbackslash}ldots from which follows Kirchhoff' s current law:

\texttt{\textbackslash}begin\texttt{\{}equation\texttt{\}}

  \qquad \texttt{\textbackslash}sum\texttt{\_}\texttt{\{}k=1\texttt{\}}\texttt{~\^}\texttt{\{}n\texttt{\}}  I\texttt{\_}k=0 \texttt{\textbackslash};

\texttt{\textbackslash}end\texttt{\{}equation\texttt{\}}
\\

Kirchhoff' s voltage law can be derives \texttt{\textbackslash}ldots
\\

\texttt{\%} Example 3

\texttt{\textbackslash}ldots which has several advantages.

\texttt{\textbackslash}begin\texttt{\{}equation\texttt{\}}

   \qquad I\texttt{\_}D = I\texttt{\_}F - I\texttt{\_}R

\texttt{\textbackslash}end\texttt{\{}equation\texttt{\}}

is the core of a very different transistor model. \texttt{\textbackslash}ldots
\section{断行和断页}
\subsection{合理分段}
\begin{itemize}
  \item \texttt{\textbackslash}\texttt{\textbackslash} or \texttt{\textbackslash}newline:

  断行但不是另起一段。\texttt{\textbackslash}\texttt{\textbackslash}也在表格、公式等地方用于分行,而
\texttt{\textbackslash}newline只用于文本段落中。
  \item \texttt{\textbackslash}\texttt{\textbackslash}\texttt{$\ast$}:

  断行,但不另起一页和不断页。
  \item \texttt{\textbackslash}newpage or \texttt{\textbackslash}clearpage:

  通常情况下两个命令都能起到另起一页的作用,但有一些区别:一是在双排版中 \texttt{\textbackslash}newpage 只起到另起一栏的作用;二是涉及到浮动体的排版上行为不同。

  \item \texttt{\textbackslash}linebreak\texttt{[}$\langle$n$\rangle$\texttt{]} \texttt{\textbackslash}nolinebreak\texttt{[}$\langle$n$\rangle$\texttt{]} \texttt{\textbackslash}pagebreak\texttt{[}$\langle$n$\rangle$\texttt{]}
  \texttt{\textbackslash}nopagebreak\texttt{[}$\langle$n$\rangle$\texttt{]}:

  不满足于 \LaTeX 默认的断行和断页位置,用其高速哪些地方适合断页,哪些地方不合适。$\langle$n$\rangle$代表合适/不合适的程度,取值范围为 0-4,不带可选参数时,缺省为4。以上命令适合给出优先考虑断行断页\texttt{\textbackslash}禁止断行断页的位置,
  但不适合直接拿来断行或断页,使用\texttt{\textbackslash}newline 或\texttt{\textbackslash}newpage 等是更好的选择。
\end{itemize}
\subsection{连字符}
如果遇到很长的英文单词,仅在单词之间的位置断行无法生成宽度匀称的行时,就要考虑从单词中间段开。对于绝大部分单词,\LaTeX 能够找到合适的断词位置,在断开的行尾加上连字符 - 。如果一些单词没能自动断词,我们可以在单词内手动使用\texttt{\textbackslash}-命令指定断词的位置。
另外,也可以使用\texttt{\textbackslash}hyphenation\texttt{\{}word list\texttt{\}}命令来指定使用连字符的位置,例如texttt{\textbackslash}hyphenation\texttt{\{}FORTRAN Hy-phen-a-tion\texttt{\}},其中的word list是不区分大小写的。
\begin{itemize}
   \item \texttt{\textbackslash}mbox\texttt{\{}text\texttt{\}}或\texttt{\textbackslash}fbox\texttt{\{}text\texttt{\}}:

   会避免text被连字符分开。\texttt{\textbackslash}fbox\texttt{\textbackslash}mbox多了个可见的框。
\end{itemize}
\section{预定义好的字符串}
\begin{itemize}
  \item \texttt{\textbackslash}today : 打印当天日期
  \item \texttt{\textbackslash}TeX : \TeX
  \item \texttt{\textbackslash}LaTeX : \LaTeX
  \item \texttt{\textbackslash}LaTeXe : \LaTeXe
\end{itemize}
\section{特殊字符和符号}
\subsection{引号}
\begin{itemize}
  \item 双引号:``...tex...''
  \item 单引号:`...tex...'

     ``Please press the `x' key.''
\end{itemize}
\subsection{短划线和连字符}
在\LaTeX 中有下面四种横杠:
\begin{itemize}
   \item - :- 连字符,用于连接词语。

   daughter-in-law, X-rated
   \item - - :-- 短破折号,常用于连接数字表示起止范围。

   pages 13--67
   \item - - - :--- 长破折号,常用于表示意思的转换。

   yes---or no?
   \item \$-\$ :$-$ 减号

   $0$,$1$ and $-1$
\end{itemize}
\subsection{波浪线}
\begin{itemize}
    \item \texttt{\textbackslash}\texttt{~\~}\texttt{\{}\texttt{\}} :\~{}

http://www.rich.edu/\~{}bush
    \item \$\texttt{\textbackslash}sim\$ : $\sim$

http://www.clever.edu/$\sim$demo
\end{itemize}
\subsection{斜杠}
\begin{itemize}
  \item read/write: 不允许用连字符拆分
  \item read\texttt{\textbackslash}slash write: 不允许用连字符拆分
\end{itemize}
\subsection{度}
\begin{itemize}
  \item \$30\texttt{\textbackslash}, \texttt{~\^}\texttt{\{}\texttt{\textbackslash}cric\texttt{\}}\texttt{\textbackslash}mathrm\texttt{\{}C\texttt{\}}\$ :
$-30\,^{\circ}\mathrm{C}$
  \item 30 \texttt{\textbackslash}textcelsius\texttt{\{}\texttt{\}}: 30 \textcelsius{}
  \item 86 \texttt{\textbackslash}textdegree\texttt{\{}\texttt{\}}F: 86 \textdegree{}F
\end{itemize}
\subsection{欧元符号}
 首先需要在导言区加载textcomp包:

 \texttt{\textbackslash}usepackage\texttt{\{}textcomp\texttt{\}}

 使用命令输出:

 \texttt{\textbackslash}texteuro

 如果所用的字体不包含欧元符号或者想用别的字体的欧元符号,导入eurosym宏包,用gen来替换official参数可以使用和当前字体匹配的欧元符号:

 \texttt{\textbackslash}usepackage[official]\texttt{\{}eurosym\texttt{\}}
 \begin{itemize}
  \item \texttt{\textbackslash}texteuro:\texteuro
  \item \texttt{\textbackslash}euro: \euro
\end{itemize}
\subsection{省略号}
\LaTeX 提供了命令\texttt{\textbackslash}ldots来生成省略号,相对于直接输入三个点的方式更为合理。\texttt{\textbackslash}ldots和\texttt{\textbackslash}dots是两个等效的命令。
\begin{itemize}
  \item ...:...
  \item \texttt{\textbackslash}ldots:\ldots
  \item \texttt{\textbackslash}dots:\dots
\end{itemize}
\subsection{连字}
有些相邻的字母在排版时会连接起来,可以通过 \texttt{\textbackslash}mbox{} 命令避免它们相连。
\begin{itemize}
  \item f\mbox{}fshf\mbox{}ilf\mbox{}luf\mbox{}f\mbox{}ia:ffshfilfluffia
  \item f \texttt{\textbackslash}mbox\texttt{\{}\texttt{\}} fshf \texttt{\textbackslash}mbox\texttt{\{}\texttt{\}} ilf \texttt{\textbackslash}mbox\texttt{\{}\texttt{\}}  luf \texttt{\textbackslash}mbox\texttt{\{}\texttt{\}}
  f \texttt{\textbackslash}mbox\texttt{\{}\texttt{\}} ia
\end{itemize}
\subsection{重音符号和特殊符号}
\LaTeX 支持用命令输入各种西欧语言的特殊符号和重音,重音符号和特殊符号命令列表:
\begin{quote}
  \`o  \qquad \'o  \qquad \^o  \qquad \~o \\
  \=o  \qquad \.o  \qquad \"o  \qquad \c c \\
  \u o \qquad \v o \qquad \H o \qquad \c o \\
  \d o \qquad \b o \qquad \t oo \\
  \oe  \qquad \OE  \qquad \ae  \qquad \AE \\
  \aa  \qquad \AA \\
  \o   \qquad  \O  \qquad \l  ~\qquad \L \\
  \i   ~\qquad  \j  ~\qquad  !`  ~\qquad ?` \\
\end{quote}

\section{国际语言支持/中文排版支持}
\LaTeX 对其他很多语言提供了支持。babel宏包可以用于对各种语言进行适配。排版中文有两种方式,一种是使用xeCJK宏包,另一种是使用 \CTeX 宏包和文档类。\CTeX 宏包和文档类是对 CJK 和 xeCJK 等宏包的进一步封装。文档类包括 \texttt{ctexart}、
ctexrep、ctexbook,分别是对 \LaTeX 的三个标准文档类 article、report、
book 的封装,对 \LaTeX 的排版样式做了许多调整,以切合中文排版风格。最新版本的 \CTeX 宏包/文档类甚至支持自
动配置字体。
\subsection{\CTeX 的安装}
   \CTeX 宏集依赖的宏包和宏集已被最常见的 \TeX 发行版 \TeX Live 和 MiK\TeX 所收录。如果本地安装的 \TeX Live 或
   MiK\TeX 不是完全版本,就需要通过这两个发行版提供的宏包管理器来安装宏包。

   \TeX Live 的宏包管理器是 tlmgr。在Linux系统上,一般需要 sudo权限才能正确地执行
   tlmgr的功能。

直接使用 sudo tlmgr [arg]时,可能会提示找不到 tlmgr 或没有这个命令。sudo 有一种内置的保护机制,只会使用安全的环境变量 PATH。如果 \TeX Live 的路径不在
sudo 的安全环境变量内,它就找不到相关的命令。可以在终端执行sudo gedit /etc/sudoers,然后将\TeX Live的路径添加到 sudo 的
secure\_path 中。不同路径用 \texttt{:}隔开。

问题解决后,在终端中依次执行以下命令,以更新 tlmgr 宏包管
理器、已安装的所有宏包、安装 \CTeX 宏集。

    sudo tlmgr update --self

    sudo tlmgr update --all

    sudo tlmgr install ctex
\subsection{\CTeX 文档类}
\CTeX 宏集提供了四个中文文档类:ctexart、ctexrep、ctexbook 和
ctexbeamer,分别对应 \LaTeX 的标准文档类 article、report、book 和
beamer。使用它们的时候,需要将涉及到的所有源文件使用 UTF-8 编码保存。

下面是使用 ctexart 文档类编写的一个例子:

    \texttt{\textbackslash}documentclass[UTF8]\texttt{\{}ctexart\texttt{\}}

    \texttt{\textbackslash}begin\texttt{\{}document\texttt{\}}

    \texttt{\textbackslash}end\texttt{\{}document\texttt{\}}
\section{单词间的空格}
\LaTeX 默认句子以句点、问号或者感叹号结尾。但是如果句点跟在一个大写字母后面,它不会认为这是句子结尾,因为大写字母后面跟句点往往是缩略词。

用户可以通过具体的命令来改变上面的默认设定。一个斜杠跟一个空格会产生一个不会被扩大的空格;一个波浪线(\texttt{$\sim$})
会产生一个既不能被扩大、也不能从这里断行的空格;在句点前使用 \texttt{\textbackslash}@ 命令,不管这个句点是不是跟在大
写字母后面,都会指定这个句子到句点就结束。使用 \texttt{\textbackslash}frenchspacing 命令可以强制不在一个句子后面插
入多余的空格。如果使用 \texttt{\textbackslash}frenchspacing 命令就没必要再用 \texttt{\textbackslash}@ 了。

例子:

Mr.~Smith was happy to see her

cf.~Fig.~5

I like BASIC\@. What about you?
\section{标题、章、节}
文档类中的几种分层次结构命令:

\texttt{\textbackslash}section\texttt{\{}...\texttt{\}}

\texttt{\textbackslash}subsection\texttt{\{}...\texttt{\}}

\texttt{\textbackslash}subsusection\texttt{\{}...\texttt{\}}

\texttt{\textbackslash}paragraph\texttt{\{}...\texttt{\}}

\texttt{\textbackslash}subparagraph\texttt{\{}...\texttt{\}}

如果你想把你的文件分成不同的部分而不影响章节编号的的使用:

\texttt{\textbackslash}part\texttt{\{}...\texttt{\}}

当使用 report 和 book 类时:

\texttt{\textbackslash}chapter\texttt{\{}...\texttt{\}}

下面是两个比较特殊的情况:
\begin{itemize}
  \item \texttt{\textbackslash}part 命令不会影响 chapter 或 section 的编号。
  \item \texttt{\textbackslash}appendix 命令没有任何参数,会把 chapter(对于 report、book)或 section(对于article)
  的数字编号转换成字母编号。
\end{itemize}

\texttt{\textbackslash}tableofcontents 命令可以用于建立目录,目录就会在这条命令所在的位置生成。一般新写的文档需要编译两次才能正确生成目录,
必要的时候\LaTeX 也会提示需要编译三次。

以上的分章节的命令都有一个可以在命令名称后加一个星号的版本,例如\texttt{\textbackslash}section\texttt{\{}...\texttt{\}}命令,
加星号之后的命令为\texttt{\textbackslash}section\texttt{$*$}\texttt{\{}...\texttt{\}},不同之处是加星版本的章节命令对应的标题不会显示在目录里,
也不会被编号。

有时候章节标题太长,这回导致其在目录里显示不佳。可以通过下面的命令在真真的标题前选择添加一个参数,指定在目录中显示的标题。
\\

\texttt{\textbackslash}chapter\texttt{[}Title for the table of contents\texttt{]}\texttt{\{}A long and espexially boring title, show
in the text\texttt{\}}
\\

整个文档的标题是通过\texttt{\textbackslash}maketitle命令产生。在调用\texttt{\textbackslash}maketitle命令之前文档标题的内容
需要由\texttt{\textbackslash}title\texttt{\{}...\texttt{\}}、\texttt{\textbackslash}author\texttt{\{}...\texttt{\}}、\texttt{\textbackslash}date\texttt{\{}...\texttt{\}}
(可选)等参数指定。在\texttt{\textbackslash}author\texttt{\{}...\texttt{\}}命令的参数中,可以用\texttt{\textbackslash}and来间隔多个作者名字。

\LaTeXe 在book文档有以下三个额外的命令,可以进行前沿、正文、后记的结构划分。这三个命令还可以和\texttt{\textbackslash}appendix命令结合,生成有前沿、正文、附录、后记四部分的文档。
\begin{itemize}
 \item \texttt{\textbackslash}frontmatter 前言部分,放置在文档主体的最开始(\texttt{\textbackslash}begin{$*$}\texttt{\{}document\texttt{\}}),
 他会把页码变成罗马数字,其后的\texttt{\textbackslash}chapter不编号
 \item \texttt{\textbackslash}mainmatter 正文部分,页码为阿拉伯数字格式,从1开始计数,其后的章节编号正常
 \item \texttt{\textbackslash}backmatter 后记部分,页码格式不变,继续正常计数;其后的\texttt{\textbackslash}chapter不编号
\end{itemize}
\section{交叉引用}
当需要对图片表格等进行引用时,用下列命令:
\begin{itemize}
  \item \texttt{\textbackslash}label\texttt{\{}marher\texttt{\}}
  \item \texttt{\textbackslash}ref\texttt{\{}marher\texttt{\}}
  \item \texttt{\textbackslash}pageref\texttt{\{}marher\texttt{\}}
\end{itemize}
其中,maeker是由用户自行定义的标识符。
\\

A refernce to this subsection
\label{sec:this} look like:
``see section~\ref{sec:this} on
page~\pageref{sec:this}."
\section{脚注}
可以使用\texttt{\textbackslash}footnote\texttt{\{}...\texttt{\}}命令来添加脚注,脚注应该紧跟在它注解的词或句子(包括标点符号)后面。

由于脚注会分散读者的注意力,所以尽量在文章主体说清楚,少用脚注。
\\

Footnotes\footnote{This is a footnote.} are often used by people using \LaTeX.
\section{强调}
在\LaTeX 中可以通过\texttt{\textbackslash}underline\texttt{\{}...\texttt{\}}命令来实现。但在印刷书籍中,一般通过
\texttt{\textbackslash}emph\texttt{\{}...\texttt{\}}命令,使用意大利字体进行强调。但并不是绝对的,需要结合具体语境。
\\

\emph{If you use emphasized text, then \LaTeX{} uses the \emph{normal} font for emphasizing.}














\end{document}
